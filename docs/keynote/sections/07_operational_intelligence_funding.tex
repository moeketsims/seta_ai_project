\section{Operational Intelligence: Aligning Support and Funding with Moments that Matter}

Institutional excellence is not only about insight; it is about \emph{orchestration}. Operational intelligence connects weekly analytics to the movement of resources and responsibilities. The system functions as a gentle conductor: surfacing where attention will have the most impact, aligning multiple actors, and generating an auditable trail of action.

\subsection{From Signals to Routing}

When the platform identifies a learner or a cluster at elevated risk, it proposes routes: teacher-led small-group instruction, supplemental instruction sessions, tutoring referrals, advisor check-ins, or brief peer-led study circles. The \emph{Interventions Queue} ranks these by expected impact and urgency. Advisors see a filtered view that prioritizes non-instructional supports (attendance nudges, schedule planning, financial aid questions), while teachers see instructional routes.

\subsection{Evidence-Linked Funding}

Funding is most effective when it is timely and targeted. For cohorts where small bursaries or tutoring hours are available, the system can suggest micro-support triggers: for example, one hour of tutoring for learners who have experienced a recent rise in risk but showed improvement after prior intervention, or a travel stipend for a study group in a specific week when mastery on a key topic is fragile. The goal is not to expand funding; it is to \emph{optimize} it, linking disbursements to evidence of need and to windows when support is most likely to change outcomes.

\subsection{Equity as a Constraint, not a Footnote}

Equity is built into routing and evaluation. Disaggregation is applied to every core metric (risk share, time-to-intervention, mastery lifts, retention), and the system flags when an otherwise effective action is widening gaps. This prompts deliberate adjustments: targeting additional representation-rich sessions to under-served groups, or redirecting tutoring hours to maximize gap closure while maintaining overall gains.

\subsection{Auditability and Learning}

Operational intelligence requires memory. Each suggested route, action taken, and outcome is logged in a human-readable way. Over time, this builds institutional wisdom: which combinations of representations, timing, and support tend to work for particular misconceptions or skills; how resource constraints shape choices; and how the ecology of support evolves across terms. Leaders and IR teams can then make policy---and budget---based on cumulative evidence, not on anecdotes.

