\section{Making Mathematics Visible: Multiple Representations and Virtual Manipulatives}

Educational value is realized not when we merely identify an error, but when we enable a learner to reconstruct a concept. The platform pairs analytics with a representation suite that brings core mathematical ideas to life in accessible ways. These are not decorative visuals; they are cognitive tools that align with well-established pedagogical principles: concreteness fading, dual coding, and variation theory.

\subsection{Number Lines, Fraction Bars, and Conceptual Anchors}

When learners miscompare fractions, the remedy begins with anchoring the idea of ``one whole'' and partitioning it into equal parts. A \emph{Visual Fraction Bar} presents a whole subdivided, allowing immediate, non-linguistic comparison of $\tfrac{1}{4}$ and $\tfrac{1}{8}$. The \emph{Number Line} complements this by placing those quantities on a single continuum: learners see that points corresponding to $\tfrac{1}{4}$ and $\tfrac{1}{8}$ sit at different distances from zero and from each other. Together, these representations construct an intuition that counters denominator magnitude bias without a lecture.

The representation suite is present across topics: integers and directed quantities on a number line; visual arrays for multiplication; area models for fraction addition; strip diagrams for ratios; and simplified algebra tiles for linear equations. Teachers can select representations that match current diagnostic needs, while learners encounter multiple ways of seeing the same idea.

\subsection{Virtual Manipulatives: Interaction as Understanding}

Interaction deepens cognition. A \emph{Virtual Manipulative Panel} gives learners draggable counters, base-ten blocks, fraction tiles, and algebra tiles, depending on the concept at hand. Instructors use these manipulatives to pose tasks like: ``Build $\tfrac{3}{4}$ in two different ways,'' or ``Represent the solution to $x+5=9$ and explain why it works.'' The goal is not gamification for its own sake; it is to make structures visible and to externalize the steps of reasoning so that feedback can be precise.

These manipulatives are integrated into the assessment and practice experience. A learner who answers a question can adjust the manipulative and then articulate, in writing or orally, how the representation supports the answer. Teachers see the state of the manipulative at submission, revealing not only \emph{what} the learner answered but \emph{how} they thought. This is invaluable for diagnosing process and giving feedback.

\subsection{Culturally Resonant Story Contexts}

Mathematics is everywhere in daily life, but context matters. The platform incorporates \emph{Story Context Cards} that frame tasks in locally meaningful scenarios: comparing airtime bundles, sharing food at a community event, budgeting for transport, reading meter readings, or analyzing a small entrepreneur's stock. These contexts are not afterthoughts; they are designed to strengthen language comprehension, bridge from informal to formal reasoning, and dignify learners' lived experiences. When learners see themselves and their communities in problems, persistence and engagement increase, and misunderstandings rooted in language are easier to surface and address.

\subsection{Accessibility and Inclusion}

The representation suite is built with accessibility in mind: high-contrast modes, descriptive labels for screen readers, and keyboard navigation for interactive components. This is not a technical boast; it is a pedagogical necessity. Inclusion is both moral and practical: without it, analytics are partial and interventions are inequitable.

