\section{Conclusion: One Cohort, One Pathway, Now}

The platform described in this keynote is not a promise of miracles; it is a set of disciplined practices made possible by timely analytics and thoughtful pedagogy. It treats diagnosis as a weekly ritual rather than an annual event. It treats misconceptions as opportunities to teach ideas, not deficiencies to lament. It treats personalization as a careful sequencing of experiences, not as atomization of the classroom. And it treats equity as a design constraint, not an afterword.

The call to action is intentionally small and immediate: choose one cohort, one high-leverage skill, and one representation-rich pathway to run for a term. Measure time-to-intervention, mastery lifts on targeted skills, and changes in risk trajectories. Track disaggregated impacts. Share reflections. Then expand.

From crisis to capability is not a slogan. It is a movement toward an institutional habit: diagnose early, act quickly, learn rigorously, and care visibly. Mathematics, taught and supported in this way, becomes both a tool for thinking and a signal of belonging. That is excellence worthy of pursuit.

