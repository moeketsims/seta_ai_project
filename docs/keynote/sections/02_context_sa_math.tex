\section{Why Mathematics, Why Now: The South African Context}

Mathematics is not merely a subject; it is a language of reasoning that enables learners to navigate finance, technology, and the sciences. In South Africa, mathematics functions as both opportunity and filter. Within the CAPS framework, mathematics outcomes benchmark access to further study and training. Yet disparities in prior opportunity, language of learning and teaching (LOLT), and resource constraints create uneven starting points for learners across quintiles and regions. A capability-informed approach does not pathologize learners or institutions; it surfaces constraints and opportunities, turning data into action and action into learning.

Several contextual realities shape the system presented here:

\begin{itemize}
  \item \textbf{Heterogeneous Prior Knowledge.} First-year cohorts (and FET cohorts in TVET pathways) enter with wide variance in fraction sense, operations fluency, and proportional reasoning. This variance is not a deficit; it is a planning input. The system recognizes heterogeneity by measuring skill-by-skill status and growth, not a single aggregate score.
  \item \textbf{Language and Representation.} Learners often confront double translation: from conversational language to academic language, and from academic language to mathematical representation. The platform addresses this through story contexts that resonate locally and through visual representations that abstract without alienating.
  \item \textbf{Resource and Time Constraints.} Teaching loads are heavy; student support centers are stretched; advisors need to prioritize. Analytics that are timely and interpretable can conserve scarce instructional time by directing attention to the highest-return tasks each week.
  \item \textbf{Equity and Ethics.} Data use is governed not only by law but by values. Predictive signals must be used to widen opportunity, not to stigmatize. Disaggregation is a tool for targeting support and closing gaps, not for labeling students. The system therefore couples risk detection with required human review and transparent action logs.
\end{itemize}

Within this context, the function of analytics is not surveillance but \emph{capability}. Weekly diagnostic assessment generates leading indicators: where learners stand on key skills, which misconceptions are most active, and which classes require targeted instructional focus. Institutional Research teams gain an evidence base for evaluating small, well-timed supports (such as supplemental instruction hours or micro-bursaries for tutoring) and for monitoring equity gaps alongside overall gains.

The rest of the keynote details how the system operationalizes these commitments: the structure of diagnostics, the logic of misconception analysis, the pedagogy of multiple representations, the assignment of personalized pathways, and the orchestration of human action informed by analytics.

