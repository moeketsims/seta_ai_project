\documentclass[12pt]{article}
\usepackage[a4paper,margin=1in]{geometry}
\usepackage[T1]{fontenc}
\usepackage[utf8]{inputenc}
\usepackage{lmodern}
\usepackage{setspace}
\usepackage{hyperref}
\usepackage{titlesec}
\usepackage{graphicx}
\usepackage{amsmath}
\usepackage{enumitem}
\usepackage{csquotes}
\usepackage{microtype}
\hypersetup{colorlinks=true,linkcolor=blue,citecolor=blue,urlcolor=blue}
\titleformat{\section}{\large\bfseries}{\thesection}{0.75em}{}
\titleformat{\subsection}{\normalsize\bfseries}{\thesubsection}{0.5em}{}
\setlist{itemsep=0.2em,topsep=0.2em}
\onehalfspacing

\title{From Crisis to Capability: Building AI that Works Where It Matters Most for Learning\\\large Towards Excellence in South African Mathematics Education through Human-Centric Analytics}
\author{Prepared for SAAIR}
\date{\today}

\begin{document}
\maketitle
\begin{abstract}
This keynote articulates an educationally grounded, analytics-led approach to improving mathematics outcomes in South Africa. It focuses on how timely diagnostics, misconception analysis, and personalized learning pathways can be translated into institutional practice through human-centric analytics and operational intelligence. Rather than describing technical implementations, it centers on instructional value, equity, and evidence-based decision-making. The system described here---a mathematics teacher assistant and analytics environment---demonstrates how weekly diagnostic signals, error analyses, and classroom-facing visual representations (including virtual manipulatives and multiple representations) are converted into earlier support, improved mastery, and more effective allocation of scarce resources. The narrative is organized to resonate with Institutional Research (IR), Teaching and Learning, Quality Assurance, and Funding stakeholders, aligning with the SAAIR theme and the South African context.
\end{abstract}

\tableofcontents
\newpage

\section{From Crisis to Capability: An Educational Promise}

South African higher education and training is living with a stubborn paradox. On the one hand, the aspirations of learners, families, and communities are anchored in mathematics---as a gateway to STEM degrees, to artisan pathways, to economic mobility and dignity. On the other, mathematics remains one of the most persistent blockers of progression, graduation, and employability. The consequence is visible in classrooms, student support centers, institutional dashboards, and national planning: too many learners become invisible to the system until it is too late for timely support.

This keynote reframes that paradox as a tractable problem of capability. To move from crisis to capability, institutions need leading indicators that reveal where understanding is developing, where it is stuck, and why. The system discussed here delivers exactly that: weekly diagnostics, misconception analysis, and personalized learning pathways for mathematics, embedded in an analytics environment that helps teachers, advisors, and leaders act in time. Its design mantra is simple: human-centric, classroom-first, and evidence-led. Instead of talking about servers and code, we will speak about understanding, time, judgment, and equity.

The promise is threefold. First, to teachers: provide formative insight that helps you target instruction to the highest-return ideas each week. Second, to students: offer experiences that meet you where you are, make mathematics sensible, and build confidence through success. Third, to leaders and IR professionals: turn operational intelligence into measurable improvements---reduced time-to-intervention, increased mastery, better retention, and smarter allocation of scarce funding support. These are not abstract aspirations; they are made possible by a workflow that begins with diagnostic assessment and culminates in a feedback loop that learns what works, for whom, under what conditions.

The platform’s analytics align to this workflow. It surfaces risk trajectories over weeks, not just end-of-term snapshots. It clusters errors into named misconceptions so that teaching time is directed to the underlying ideas rather than surface symptoms. It maintains skill maps and pathways so that progress is not a binary pass/fail but a growing web of mastery. And it pairs its analytics with visual, interactive mathematics experiences that make concepts concrete: number lines, fraction bars, and virtual manipulatives rendered alongside questions, with culturally relevant story contexts that speak to the realities of South African learners.

This is a keynote about educational value: what teachers can do on Monday, what students can understand by Wednesday, and what leaders can measure by Friday. The analytics are timely, interpretable, and consequential. And the entire approach is situated within South African policy and practice: the CAPS-aligned curriculum, the roles of Institutional Research and Quality Assurance, the funding pressures that demand cost-effective interventions, and the ethical commitments required by law and by conscience.


\section{Why Mathematics, Why Now: The South African Context}

Mathematics is not merely a subject; it is a language of reasoning that enables learners to navigate finance, technology, and the sciences. In South Africa, mathematics functions as both opportunity and filter. Within the CAPS framework, mathematics outcomes benchmark access to further study and training. Yet disparities in prior opportunity, language of learning and teaching (LOLT), and resource constraints create uneven starting points for learners across quintiles and regions. A capability-informed approach does not pathologize learners or institutions; it surfaces constraints and opportunities, turning data into action and action into learning.

Several contextual realities shape the system presented here:

\begin{itemize}
  \item \textbf{Heterogeneous Prior Knowledge.} First-year cohorts (and FET cohorts in TVET pathways) enter with wide variance in fraction sense, operations fluency, and proportional reasoning. This variance is not a deficit; it is a planning input. The system recognizes heterogeneity by measuring skill-by-skill status and growth, not a single aggregate score.
  \item \textbf{Language and Representation.} Learners often confront double translation: from conversational language to academic language, and from academic language to mathematical representation. The platform addresses this through story contexts that resonate locally and through visual representations that abstract without alienating.
  \item \textbf{Resource and Time Constraints.} Teaching loads are heavy; student support centers are stretched; advisors need to prioritize. Analytics that are timely and interpretable can conserve scarce instructional time by directing attention to the highest-return tasks each week.
  \item \textbf{Equity and Ethics.} Data use is governed not only by law but by values. Predictive signals must be used to widen opportunity, not to stigmatize. Disaggregation is a tool for targeting support and closing gaps, not for labeling students. The system therefore couples risk detection with required human review and transparent action logs.
\end{itemize}

Within this context, the function of analytics is not surveillance but \emph{capability}. Weekly diagnostic assessment generates leading indicators: where learners stand on key skills, which misconceptions are most active, and which classes require targeted instructional focus. Institutional Research teams gain an evidence base for evaluating small, well-timed supports (such as supplemental instruction hours or micro-bursaries for tutoring) and for monitoring equity gaps alongside overall gains.

The rest of the keynote details how the system operationalizes these commitments: the structure of diagnostics, the logic of misconception analysis, the pedagogy of multiple representations, the assignment of personalized pathways, and the orchestration of human action informed by analytics.


\section{Diagnostics that Matter: From Errors to Explanations}

At the heart of the platform is a weekly diagnostic that is short, focused, and instructionally actionable. Each diagnostic targets a small set of high-leverage skills (for example: comparing fractions with unlike denominators; interpreting points on a number line; solving for an unknown in a one-step equation; reading multiplicative comparisons). The aim is not high-stakes sorting but low-stakes sensing: to illuminate learning in motion.

\subsection{Design Principles for Diagnostic Items}

Diagnostic items in the platform follow three principles:

\begin{enumerate}
  \item \textbf{Elicitable Misconceptions.} Items are constructed so that common misconceptions produce predictable wrong answers. For example, when comparing $\tfrac{1}{4}$ and $\tfrac{1}{8}$, a learner who believes “larger denominator means larger value” will systematically choose $\tfrac{1}{8}$. Such items allow analytics to do more than count errors; they attribute errors to \emph{concepts}.
  \item \textbf{Representation-Rich.} Items are surrounded by supportive representations---fraction bars, number lines, manipulatives, and story contexts---that help learners reason and that give teachers visibility into where understanding breaks down. The system presents these supports not as hints to be toggled away but as part of the mathematical conversation.
  \item \textbf{Short Cycle, High Frequency.} Because the platform is built for weekly cadence, diagnostics are intentionally brief. This allows teachers to receive fresh insight without sacrificing teaching time, and it turns intervention into a rhythm rather than a scramble.
\end{enumerate}

\subsection{From Wrong Answers to Misconception Clusters}

Counting wrong answers is insufficient for instruction. The platform organizes errors into misconception clusters that are interpretable by teachers and aligned to skills. Examples include: ``denominator magnitude bias'' in fraction comparison; ``whole-number bias'' in decimal operations; ``additive overgeneralization'' in proportional reasoning; and ``operator-as-instruction'' in word problems. In the teacher analytics, a \emph{Misconception Dashboard} surfaces the top clusters by class and cohort, with plain-language descriptions and examples drawn from recent responses.

Two properties make these clusters practically useful:

\begin{itemize}
  \item \textbf{Actionability.} Each cluster links to a set of targeted activities or representations. If a class shows a spike in denominator magnitude bias, the platform proposes a short sequence using fraction bars and number lines that build the idea of unit size and partitioning. Teachers can preview the representations and select those that match their learners' needs.
  \item \textbf{Trajectory Awareness.} Clusters are not static taxonomies; they have trend lines. The \emph{Diagnostic Deep Dive} view presents how a misconception's prevalence changes over weeks, after targeted instruction, giving teachers and IR teams a way to see whether instructional responses are working.
\end{itemize}

\subsection{Interpretable Risk, Not Opaque Labels}

Risk analytics in the platform are conveyed as trajectories, not single numbers. The \emph{Predictive Risk Analytics} view shows the share of learners currently on track, in need of support, and at risk, with attention to how quickly support follows identification. Crucially, every risk signal is paired with its instructional rationale: which misconceptions or skills are driving the signal, and what actions are recommended. The point is not to predict failure; it is to prioritize support.

\subsection{Time-to-Intervention as a First-Class Metric}

To make analytics consequential, the system elevates \emph{time-to-intervention} as a core metric. The teacher dashboard displays a distribution of days from first risk signal to first support action, along with the percentage of at-risk learners who received support within seven days. This reorients practice: teachers and advisors do not just watch risk; they shorten the latency from signal to help. Institutional leaders can then measure whether investments---in supplemental instruction, tutoring, or micro-bursaries---translate into shorter latencies and better outcomes.


\section{Making Mathematics Visible: Multiple Representations and Virtual Manipulatives}

Educational value is realized not when we merely identify an error, but when we enable a learner to reconstruct a concept. The platform pairs analytics with a representation suite that brings core mathematical ideas to life in accessible ways. These are not decorative visuals; they are cognitive tools that align with well-established pedagogical principles: concreteness fading, dual coding, and variation theory.

\subsection{Number Lines, Fraction Bars, and Conceptual Anchors}

When learners miscompare fractions, the remedy begins with anchoring the idea of ``one whole'' and partitioning it into equal parts. A \emph{Visual Fraction Bar} presents a whole subdivided, allowing immediate, non-linguistic comparison of $\tfrac{1}{4}$ and $\tfrac{1}{8}$. The \emph{Number Line} complements this by placing those quantities on a single continuum: learners see that points corresponding to $\tfrac{1}{4}$ and $\tfrac{1}{8}$ sit at different distances from zero and from each other. Together, these representations construct an intuition that counters denominator magnitude bias without a lecture.

The representation suite is present across topics: integers and directed quantities on a number line; visual arrays for multiplication; area models for fraction addition; strip diagrams for ratios; and simplified algebra tiles for linear equations. Teachers can select representations that match current diagnostic needs, while learners encounter multiple ways of seeing the same idea.

\subsection{Virtual Manipulatives: Interaction as Understanding}

Interaction deepens cognition. A \emph{Virtual Manipulative Panel} gives learners draggable counters, base-ten blocks, fraction tiles, and algebra tiles, depending on the concept at hand. Instructors use these manipulatives to pose tasks like: ``Build $\tfrac{3}{4}$ in two different ways,'' or ``Represent the solution to $x+5=9$ and explain why it works.'' The goal is not gamification for its own sake; it is to make structures visible and to externalize the steps of reasoning so that feedback can be precise.

These manipulatives are integrated into the assessment and practice experience. A learner who answers a question can adjust the manipulative and then articulate, in writing or orally, how the representation supports the answer. Teachers see the state of the manipulative at submission, revealing not only \emph{what} the learner answered but \emph{how} they thought. This is invaluable for diagnosing process and giving feedback.

\subsection{Culturally Resonant Story Contexts}

Mathematics is everywhere in daily life, but context matters. The platform incorporates \emph{Story Context Cards} that frame tasks in locally meaningful scenarios: comparing airtime bundles, sharing food at a community event, budgeting for transport, reading meter readings, or analyzing a small entrepreneur's stock. These contexts are not afterthoughts; they are designed to strengthen language comprehension, bridge from informal to formal reasoning, and dignify learners' lived experiences. When learners see themselves and their communities in problems, persistence and engagement increase, and misunderstandings rooted in language are easier to surface and address.

\subsection{Accessibility and Inclusion}

The representation suite is built with accessibility in mind: high-contrast modes, descriptive labels for screen readers, and keyboard navigation for interactive components. This is not a technical boast; it is a pedagogical necessity. Inclusion is both moral and practical: without it, analytics are partial and interventions are inequitable.


\section{Personalized Learning Pathways: From Signals to Sequences}

Personalization, in this system, is not a marketing slogan; it is a disciplined mapping from diagnostic signals to well-scaffolded sequences of learning experiences. The platform maintains a skill map that captures prerequisite relationships and conceptual neighborhoods: for example, fraction comparison draws on unit fraction reasoning, partitioning, and number line fluency; solving one-step equations draws on inverse operations, balance, and properties of equality.

\subsection{Assigning Pathways}

When a learner completes the weekly diagnostic, the platform updates the learner's profile: current mastery by skill, active misconceptions, and engagement signals. A pathway generator then constructs a short, bounded sequence for the week (or fortnight), combining three elements:

\begin{enumerate}
  \item \textbf{Core Tasks.} A small set of carefully chosen items that directly address the learner's most impactful misconception or missing prerequisite. These tasks come with the representations most likely to resolve the conceptual snag.
  \item \textbf{Fluency Boosters.} Spaced practice on foundational facts or procedures (for instance, fraction equivalence or integer operations) to increase automaticity where it matters.
  \item \textbf{Application Tasks.} Short, contextualized problems that require transfer, ensuring that gains are not fragile and that learners can recognize when and how to use new understanding.
\end{enumerate}

The system's \emph{Skill Progression Map} visualizes this sequence for the learner and teacher, showing current location, next step, and why the step was chosen. Critically, the explanation is written in student-friendly language (and can be localized), so that agency and understanding remain with the learner.

\subsection{A Worked Example}

Consider a learner whose diagnostic reveals denominator magnitude bias and weak number line fluency. The assigned pathway might be: (1) compare $\tfrac{1}{4}$ and $\tfrac{1}{8}$ using fraction bars, (2) place $\tfrac{1}{8}$, $\tfrac{1}{4}$, and $\tfrac{3}{8}$ on a number line, (3) build $\tfrac{3}{4}$ from unit fractions using manipulatives, (4) solve a context problem comparing portions of mobile data bundles with different denominators. Alongside each activity, the platform provides a short rationale: ``We are using the number line to see how far each fraction is from zero; equal partitions mean smaller parts when the denominator is larger.''

\subsection{Balancing Personalization and Cohesion}

Personalization must respect classroom cohesion. The platform is intentionally conservative in fragmenting the class: teachers can group learners with similar pathways, and whole-class representation-rich mini-lessons can address the most prevalent misconceptions. The \emph{Interventions Queue} supports this by listing suggested small-group sessions and one-on-one check-ins, sorted by expected impact on class outcomes and by urgency. This makes personalization administratively feasible, preserving teacher energy for feedback and instruction.

\subsection{Motivation and Belonging}

Personalization should not isolate learners. The platform emphasizes visible progress, frequent success experiences, and reflective prompts that connect effort to understanding. Learners see growth on their \emph{Skill Progression Map}, not just a list of tasks. They are asked to explain a representation, teach a peer, or connect an idea to a story context. These practices cultivate mathematical identity and belonging---critical ingredients for retention and long-term success.

\subsection{A Four-Week Learner Journey}

To make the pathway concrete, consider a four-week arc for Nomsa, a first-year student in a STEM access programme. In Week 1, diagnostics reveal denominator magnitude bias and shaky partitioning. Her pathway pairs fraction bars with number lines and asks her to build $\tfrac{3}{4}$ from unit fractions in multiple ways. In a small-group session, the teacher uses a strip diagram to connect parts to wholes and invites learners to explain why ``more parts means smaller pieces.'' Nomsa writes a short reflection: ``I used to look at the number and think bigger is bigger. The line shows the size.''

In Week 2, the focus shifts to fraction equivalence. Nomsa uses manipulatives to generate equivalent fractions for $\tfrac{1}{4}$ and $\tfrac{3}{4}$, then places them on a number line. She solves a context problem about sharing bread at a community event, comparing $\tfrac{3}{4}$ to $\tfrac{6}{8}$. Her diagnostic shows fewer errors in comparison items, and the misconception cluster prevalence declines for her group. The teacher notes that explanation prompts helped: learners who taught a peer showed stronger gains.

Week 3 introduces operations: adding fractions with like denominators using area models, then bridging to unlike denominators by reasoning about equivalence rather than rule-chasing. Nomsa initially reverts to adding numerators and denominators, a well-known overgeneralization. The pathway addresses this with a deliberate contrast task: two problems side-by-side that look similar but behave differently, making the structure salient. By mid-week, Nomsa demonstrates with manipulatives why denominators cannot be added. Her risk trajectory dips from ``needs support'' to ``on watch''.

Week 4 consolidates transfer. Nomsa applies fraction comparisons and operations to a budgeting scenario involving mobile data and transport, selecting which deal is better and explaining how she knows. The \emph{Skill Progression Map} shows growth across three nodes, with a lingering flag on complex denominators---useful guidance for the next cycle. Importantly, Nomsa reports higher confidence and volunteers to demo the number line to peers. The pathway has not just moved scores; it has moved identity.

\subsection{Teacher Moves that Sustain Personalization}

Personalization is sustained by a handful of recurring moves: (1) \emph{Anticipate} misconceptions before instruction and select representations accordingly; (2) \emph{Elicit} thinking publicly using manipulatives and number lines, making reasoning visible; (3) \emph{Press} for connections among representations (``Where is this on the line?'' ``How does the bar show the same idea?''); (4) \emph{Name} the structure so that learners can carry it forward (``unit size,'' ``equal partitions,'' ``distance from zero''); and (5) \emph{Reflect} using brief student explanations attached to tasks. The platform scaffolds these moves by aligning pathway tasks with prompts and by giving teachers quick access to the visuals that make the ideas stick.

\section{Teacher-Facing Analytics: Interpretable, Timely, and Actionable}

Teachers are the principal audience for weekly analytics. The teacher dashboard is designed to answer three questions every week: What should I teach differently? Which learners need what kind of support? How will I know if it worked? The answer is organized into a small set of views that privilege clarity over complexity.

\subsection{Overview: Risk, Mastery, and Workload}

The overview page presents: (1) current distribution of learners by status (on track, needs support, at risk), (2) a 12-week performance trend line for the class, and (3) a ``workload'' snapshot indicating how many suggested small-group sessions and one-on-one check-ins are queued. Each of these elements is linked to the reasoning behind it: which skills and misconceptions contribute to the status, and how the recommended session would address them.

\subsection{Misconception Dashboard and Deep Dives}

The \emph{Misconception Dashboard} lists the top misconception clusters by prevalence and shows their week-over-week movement. Clicking into a cluster opens a \emph{Diagnostic Deep Dive}: a narrative page with representative learner responses (de-identified), the representations best suited to address the misconception, and a short set of teacher moves for the next lesson. The aim is to minimize the distance between insight and instruction: within a few minutes, a teacher can assemble a 20-minute representation-rich mini-lesson targeted to the class's current needs.

\subsection{Skill Heatmaps and Progress Maps}

Two complementary views support planning and feedback. A \emph{Skill Heatmap} shows mastery levels across a subset of key skills for each learner, enabling grouping decisions that vary across weeks. A \emph{Skill Progression Map} shows how conceptual neighborhoods are strengthening over time, reinforcing the idea that mathematical understanding is a network of relationships, not a checklist. Teachers can annotate these views, leaving notes about what worked in class and which representations were most effective.

\subsection{Intervention Effectiveness}

The system tracks the relationship between interventions and subsequent outcomes. The \emph{Intervention Effectiveness} view displays changes in mastery on targeted skills, changes in error patterns, and movement in overall risk categories following small-group sessions, tutoring, or advisor outreach. The intent is not to ``prove'' a causal effect in the strictest sense during weekly operations, but to give reasonable, proximate evidence that a given action is helping. For more formal evaluation, the IR playbook (Section~\ref{sec:evidence}) provides study designs and metrics.

\subsection{A Rhythm of Practice}

These analytics support a weekly rhythm: diagnose on Friday, plan on the weekend or Monday morning, teach targeted representations early in the week, offer group and individual support mid-week, and reflect on Thursday. By keeping the cycle short and the views simple, the platform respects teacher time while steadily raising the precision of instruction.


\section{Operational Intelligence: Aligning Support and Funding with Moments that Matter}

Institutional excellence is not only about insight; it is about \emph{orchestration}. Operational intelligence connects weekly analytics to the movement of resources and responsibilities. The system functions as a gentle conductor: surfacing where attention will have the most impact, aligning multiple actors, and generating an auditable trail of action.

\subsection{From Signals to Routing}

When the platform identifies a learner or a cluster at elevated risk, it proposes routes: teacher-led small-group instruction, supplemental instruction sessions, tutoring referrals, advisor check-ins, or brief peer-led study circles. The \emph{Interventions Queue} ranks these by expected impact and urgency. Advisors see a filtered view that prioritizes non-instructional supports (attendance nudges, schedule planning, financial aid questions), while teachers see instructional routes.

\subsection{Evidence-Linked Funding}

Funding is most effective when it is timely and targeted. For cohorts where small bursaries or tutoring hours are available, the system can suggest micro-support triggers: for example, one hour of tutoring for learners who have experienced a recent rise in risk but showed improvement after prior intervention, or a travel stipend for a study group in a specific week when mastery on a key topic is fragile. The goal is not to expand funding; it is to \emph{optimize} it, linking disbursements to evidence of need and to windows when support is most likely to change outcomes.

\subsection{Equity as a Constraint, not a Footnote}

Equity is built into routing and evaluation. Disaggregation is applied to every core metric (risk share, time-to-intervention, mastery lifts, retention), and the system flags when an otherwise effective action is widening gaps. This prompts deliberate adjustments: targeting additional representation-rich sessions to under-served groups, or redirecting tutoring hours to maximize gap closure while maintaining overall gains.

\subsection{Auditability and Learning}

Operational intelligence requires memory. Each suggested route, action taken, and outcome is logged in a human-readable way. Over time, this builds institutional wisdom: which combinations of representations, timing, and support tend to work for particular misconceptions or skills; how resource constraints shape choices; and how the ecology of support evolves across terms. Leaders and IR teams can then make policy---and budget---based on cumulative evidence, not on anecdotes.


\section{Evidence and Improvement: An IR Playbook}
\label{sec:evidence}

Institutional Research (IR) is essential for turning promising practices into reliable improvements. The platform therefore ships with an evaluation playbook that fits academic operations. The goal is credible, decision-grade evidence without paralyzing the work of teaching and support.

\subsection{Core Outcomes and Leading Indicators}

The playbook defines a small set of outcomes and leading indicators:
\begin{itemize}
  \item \textbf{Leading indicators} (weekly): risk share, time-to-intervention, mastery on targeted skills, prevalence of key misconception clusters, engagement signals (on-task time, completion of pathway tasks).
  \item \textbf{Intermediate outcomes} (termly): pass rates in gateway modules, credit accumulation rate, re-enrolment (retention) for the next term, movement in equity gaps.
  \item \textbf{Institutional outcomes} (annual): progression, graduation trajectory, directional movement in cost-per-graduate.
\end{itemize}

\subsection{Designs that Respect Operations}

Several quasi-experimental designs are feasible in ordinary academic practice:
\begin{enumerate}
  \item \textbf{Matched Cohorts.} Compare classes or cohorts with similar prior achievement and demographics where the platform is used versus where it is not yet used, while controlling for observable differences.
  \item \textbf{Within-Cohort Phasing.} Introduce certain features (e.g., representation-rich mini-lessons or advisor routing) at different times across sections, creating natural comparisons.
  \item \textbf{Regression Discontinuity at a Policy Threshold.} When small supports (like micro-bursaries for tutoring) are allocated based on a transparent threshold (e.g., sustained risk trajectory), compare learners near the cut-off.
\end{enumerate}

The platform’s logging of actions and timing enables IR teams to align analyses tightly with what actually happened. Disaggregation is expected, not optional, so that equity impacts are visible.

\subsection{Teacher Learning and Organisational Memory}

Evidence is not only about end-states; it is about building staff capability. Teachers can attach short reflections to actions (``the number line helped once we connected to the fraction bar''), and these become searchable snippets linked to metrics. Over time, teaching teams inherit a living repository of what tends to work for particular misconceptions in their context. IR teams can curate periodic briefs that synthesize these reflections with quantitative results.

\subsection{Publishing and Accountability}

Because the platform emphasizes interpretable analytics and ethical data use, it lowers the barrier for transparent reporting to councils, senates, and external quality bodies. A termly ``outcomes and equity'' brief can be generated that shows gains, gaps, and next steps, along with a succinct description of data governance. Accountability becomes a story of capability: what the institution is learning about how to help its students.


\section{Governance and Ethics: Privacy, Agency, and Care}

Analytics in education carry obligations. The platform’s approach to governance is designed to honor legal requirements and ethical commitments while preserving the agility needed for weekly improvement.

\subsection{Data Minimisation and Purpose Limitation}

Only data that directly serves teaching, learning, or evaluation enters the platform. Diagnostic responses, pathway completions, and action logs are central. Where possible, identifiers are abstracted in analytics views, and role-based access ensures that staff see only what they need to act in their role. Data are used to \emph{support} learners, not to brand them.

\subsection{Transparency and Student Dignity}

Risk signals are not labels. They are provisional hypotheses to be reviewed by humans. Learners can see and contest their profiles; teachers can annotate and correct. Every analytic view includes a plain-language explanation: what is being shown, why it matters, and how it will be used to help. The platform’s default language invites dialogue, not surveillance.

\subsection{Human-in-the-Loop Decisions}

No consequential decisions are automated. The system proposes; staff decide. Logs record both proposals and decisions, enabling audits and reflective practice. This not only satisfies governance; it reinforces the institutional culture of care and professional judgment.

\subsection{Equity by Design}

Disaggregation is embedded across the analytics so that gaps are never hidden by averages. Interventions are evaluated for both overall gains and gap closure. When a strategy improves averages but widens a gap, the platform flags this for attention. Equity is thus treated as a constraint in optimisation, not as an afterthought.


\section{Conclusion: One Cohort, One Pathway, Now}

The platform described in this keynote is not a promise of miracles; it is a set of disciplined practices made possible by timely analytics and thoughtful pedagogy. It treats diagnosis as a weekly ritual rather than an annual event. It treats misconceptions as opportunities to teach ideas, not deficiencies to lament. It treats personalization as a careful sequencing of experiences, not as atomization of the classroom. And it treats equity as a design constraint, not an afterword.

The call to action is intentionally small and immediate: choose one cohort, one high-leverage skill, and one representation-rich pathway to run for a term. Measure time-to-intervention, mastery lifts on targeted skills, and changes in risk trajectories. Track disaggregated impacts. Share reflections. Then expand.

From crisis to capability is not a slogan. It is a movement toward an institutional habit: diagnose early, act quickly, learn rigorously, and care visibly. Mathematics, taught and supported in this way, becomes both a tool for thinking and a signal of belonging. That is excellence worthy of pursuit.



\end{document}
